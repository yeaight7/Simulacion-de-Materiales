\documentclass[11pt,a4paper]{article}

% Paquetes necesarios
\usepackage[utf8]{inputenc}
\usepackage[T1]{fontenc}
\usepackage{lmodern}
\usepackage[english]{babel}
\usepackage{amsmath,amssymb}
\usepackage{graphicx}
\usepackage{booktabs}
\usepackage{siunitx}
\usepackage{geometry}
\usepackage{hyperref}
\usepackage{fancyhdr}
\usepackage{listings}
\usepackage{xcolor}
\usepackage{setspace}

% Configuración de geometría y siunitx
\geometry{margin=2.2cm}
\setstretch{1.25}
\sisetup{output-decimal-marker = {.}}

% Configuración de encabezados
\pagestyle{fancy}
\setlength{\headheight}{14.49998pt}
\fancyhf{}
\rhead{Simulación de Materiales}
\lhead{Trabajo de Curso}
\cfoot{\thepage}

% Configuración de código MATLAB
\lstset{
    language=Matlab,
    basicstyle=\ttfamily\small,
    keywordstyle=\color{blue},
    commentstyle=\color{green! 60!black},
    stringstyle=\color{red},
    numbers=left,
    numberstyle=\tiny\color{gray},
    stepnumber=1,
    numbersep=5pt,
    backgroundcolor=\color{white},
    showspaces=false,
    showstringspaces=false,
    showtabs=false,
    frame=single,
    tabsize=2,
    captionpos=b,
    breaklines=true,
    breakatwhitespace=false
}

\title{\huge{\textbf{Trabajo de Curso \\ Simulación de Materiales}}}
\author{\huge{Javier Rivero Iglesias}}
\date{}

\begin{document}

\maketitle
\thispagestyle{empty}

\newpage
\tableofcontents
\newpage

\section{Introducción}

Este trabajo presenta el desarrollo e implementación de un programa de análisis estructural mediante el Método de Elementos Finitos (MEF) para estructuras reticuladas planas. Se analizan dos hipótesis de comportamiento:

\begin{itemize}
    \item \textbf{Nudos biarticulados:} Solo esfuerzos axiales (2 GDL por nodo)
    \item \textbf{Nudos rígidos:} Esfuerzos axiales y momentos flectores (3 GDL por nodo)
\end{itemize}

El objetivo es comparar ambos modelos evaluando desplazamientos, tensiones y reacciones. 

\newpage
\section{Descripción del Problema}

\subsection{Geometría de la Estructura}

La estructura analizada consiste en una cercha reticulada plana compuesta por: 

\begin{itemize}
    \item \textbf{24 nodos}
    \item \textbf{51 barras}
    \item \textbf{4 apoyos fijos} ubicados en los nodos 1, 2, 8 y 9 (base de la estructura)
    \item \textbf{Longitud total: } \SI{50}{m}
    \item \textbf{Altura total:} \SI{5}{m}
\end{itemize}

La visualización gráfica detallada de la estructura, incluyendo la numeración de nodos y barras, se presenta en las \textcolor{blue}{\underline{Figuras \ref{fig:estructura_biarticulado}}} y \textcolor{blue}{\underline{\ref{fig:estructura_rigido}}} correspondientes a cada modelo analizado.

\subsection{Condiciones de Contorno}

\subsubsection{Apoyos}

Los apoyos se encuentran ubicados en la fila inferior de la estructura: 

\begin{itemize}
    \item \textbf{Nudos biarticulados:} Restricción de desplazamientos en $x$ e $y$ ($u_x = u_y = 0$)
    \item \textbf{Nudos rígidos: } Restricción de desplazamientos y rotación ($u_x = u_y = \theta = 0$)
\end{itemize}

\subsubsection{Cargas}

Se aplican cargas verticales concentradas en los nodos de la banda de rodadura (nodos 1 a 9):

\begin{table}[ht]
\centering
\label{tab:cargas}
\begin{tabular}{@{}cc@{}}
\toprule
Nodos & Fuerza vertical [N] \\
\midrule
1, 9 & $-125\,000$ \\
2, 3, 4, 5, 6, 7, 8 & $-250\,000$ \\
\midrule
\textbf{Total} & $-2\,000\,000$ \\
\bottomrule
\end{tabular}
\caption{Cargas aplicadas en la estructura.}
\end{table}

La oración \textit{"sobre la que actúa una carga vertical con sentido descendente de 2 MN, que se reparte uniformemente"} es un poco ambigua y se puede interpretar de distintas formas. Sin embargo, se ha optado por este enfoque.
La carga total de 2 MN se modela como carga uniforme sobre las barras inferiores y se discretiza en cargas nodales, resultando 125 kN en los nodos extremos y 250 kN en los interiores.

\newpage
\subsection{Propiedades de los Materiales}

La estructura está compuesta por tres tipos de materiales con las siguientes propiedades:

\begin{table}[ht]
\centering
\label{tab:materiales}
\begin{tabular}{@{}ccc@{}}
\toprule
Material & Módulo de Young $E$ [Pa] & Límite elástico $\sigma_y$ [Pa] \\
\midrule
1 & $2.00\times10^{11}$ & $2.50\times10^{8}$ \\
2 & $1.60\times10^{11}$ & $2.50\times10^{8}$ \\
3 & $1.60\times10^{11}$ & $2.10\times10^{8}$ \\
\bottomrule
\end{tabular}
\caption{Propiedades de los materiales.}
\end{table}

\subsection{Propiedades de las Barras}

Se consideran tres tipos de secciones transversales:

\begin{table}[ht]
\centering
\label{tab:secciones}
\begin{tabular}{@{}ccc@{}}
\toprule
Sección & Área $A$ [m$^2$] & Momento de inercia $I$ [m$^4$] \\
\midrule
1 & 0.0150 & $2.80\times10^{-5}$ \\
2 & 0.0144 & $1.728\times10^{-5}$ \\
3 & 0.0254 & $5.153\times10^{-5}$ \\
\bottomrule
\end{tabular}
\caption{Propiedades geométricas de las secciones.}
\end{table}

\newpage
\section{Fundamentos Teóricos}

\subsection{Método de Elementos Finitos}

El Método de Elementos Finitos (MEF) discretiza la estructura continua en elementos finitos conectados por nodos. Para cada elemento se establece una relación entre fuerzas nodales y desplazamientos mediante la matriz de rigidez: 

\begin{equation}
\mathbf{f} = \mathbf{K} \mathbf{u}
\end{equation}

donde $\mathbf{f}$ es el vector de fuerzas nodales, $\mathbf{K}$ la matriz de rigidez y $\mathbf{u}$ el vector de desplazamientos.

\subsection{Modelo de Nudos Biarticulados}

En este modelo, cada nodo tiene 2 grados de libertad ($u_x$, $u_y$). La matriz de rigidez local de un elemento tipo barra es:

\begin{equation}
\mathbf{k}^{local} = \frac{EA}{L}
\begin{bmatrix}
1 & 0 & -1 & 0 \\
0 & 0 & 0 & 0 \\
-1 & 0 & 1 & 0 \\
0 & 0 & 0 & 0
\end{bmatrix}
\end{equation}

donde $E$ es el módulo de Young, $A$ el área de la sección transversal y $L$ la longitud del elemento.

La transformación al sistema global se realiza mediante la matriz de rotación $\mathbf{T}$:

\begin{equation}
\mathbf{T} = 
\begin{bmatrix}
\cos\alpha & \sin\alpha & 0 & 0 \\
-\sin\alpha & \cos\alpha & 0 & 0 \\
0 & 0 & \cos\alpha & \sin\alpha \\
0 & 0 & -\sin\alpha & \cos\alpha
\end{bmatrix}
\end{equation}

siendo $\alpha$ el ángulo que forma el elemento con el eje horizontal.

La matriz de rigidez global del elemento es:

\begin{equation}
\mathbf{K}^{global} = \mathbf{T}^T \mathbf{k}^{local} \mathbf{T}
\end{equation}

La tensión axial en cada barra se calcula como:

\begin{equation}
\sigma = E \frac{\Delta L}{L}
\end{equation}

donde $\Delta L$ es el alargamiento o acortamiento del elemento.

\subsection{Modelo de Nudos Rígidos}

En este modelo, cada nodo tiene 3 grados de libertad ($u_x$, $u_y$, $\theta$):

\begin{equation}
\mathbf{k}^{local} = 
\begin{bmatrix}
\frac{EA}{L} & 0 & 0 & -\frac{EA}{L} & 0 & 0 \\
0 & \frac{12EI}{L^3} & \frac{6EI}{L^2} & 0 & -\frac{12EI}{L^3} & \frac{6EI}{L^2} \\
0 & \frac{6EI}{L^2} & \frac{4EI}{L} & 0 & -\frac{6EI}{L^2} & \frac{2EI}{L} \\
-\frac{EA}{L} & 0 & 0 & \frac{EA}{L} & 0 & 0 \\
0 & -\frac{12EI}{L^3} & -\frac{6EI}{L^2} & 0 & \frac{12EI}{L^3} & -\frac{6EI}{L^2} \\
0 & \frac{6EI}{L^2} & \frac{2EI}{L} & 0 & -\frac{6EI}{L^2} & \frac{4EI}{L}
\end{bmatrix}
\end{equation}

donde $I$ es el momento de inercia de la sección transversal.

La tensión total en cada barra considera tanto el esfuerzo axial como el flector:

\begin{equation}
\sigma_{total} = \sigma_{axial} + \sigma_{flexión} = E\frac{\Delta L}{L} + \frac{M \cdot c}{I}
\end{equation}

donde $M$ es el momento flector máximo y $c$ es la distancia desde el eje neutro a la fibra extrema.

\subsection{Ensamblaje del Sistema Global}

El sistema global de ecuaciones se expresa como:

\begin{equation}
\begin{bmatrix}
\mathbf{S}_{pp} & \mathbf{S}_{pd} \\
\mathbf{S}_{dp} & \mathbf{S}_{dd}
\end{bmatrix}
\begin{bmatrix}
\mathbf{d}_p \\
\mathbf{d}_d
\end{bmatrix}
=
\begin{bmatrix}
\mathbf{p}_p \\
\mathbf{p}_d
\end{bmatrix}
\end{equation}

donde el subíndice $p$ denota grados de libertad prescritos (apoyos) y $d$ los grados de libertad libres.

Los desplazamientos libres se obtienen resolviendo: 

\begin{equation}
\mathbf{d}_d = \mathbf{S}_{dd}^{-1} (\mathbf{p}_d - \mathbf{S}_{dp} \mathbf{d}_p)
\end{equation}

Las reacciones en los apoyos se calculan como:

\begin{equation}
\mathbf{p}_p = \mathbf{S}_{pp} \mathbf{d}_p + \mathbf{S}_{pd} \mathbf{d}_d
\end{equation}

\newpage
\section{Implementación Computacional}

\subsection{Arquitectura del Programa}

\begin{enumerate}
    \item \textbf{Lectura de datos} (\texttt{read\_input.m})
    \item \textbf{Preprocesamiento} (\texttt{preprocessor.m})
    \item \textbf{Procesamiento} (\texttt{processor.m})
    \item \textbf{Postprocesamiento} (\texttt{postprocessor.m})
\end{enumerate}

El flujo de ejecución del programa sigue la secuencia:
\begin{center}
\texttt{Input} $\rightarrow$ \texttt{Lectura} $\rightarrow$ \texttt{Preprocesamiento} $\rightarrow$ \texttt{Procesamiento} $\rightarrow$ \texttt{Postprocesamiento} $\rightarrow$ \texttt{Output}
\end{center}

\subsection{Módulo de Lectura de Datos}

Este módulo (\texttt{read\_input.m}) lee el archivo de texto \texttt{estructura.txt} que contiene: 

\begin{itemize}
    \item Coordenadas de los nodos
    \item Definición de apoyos
    \item Propiedades de materiales
    \item Propiedades de secciones
    \item Conectividad de elementos (barras)
    \item Cargas aplicadas
\end{itemize}

La función devuelve una estructura \texttt{PROB} con todos los datos organizados para su posterior procesamiento.

\subsection{Módulo de Preprocesamiento}

El preprocesamiento realiza las siguientes tareas:

\begin{enumerate}
    \item Cálculo de matrices de rigidez locales para cada elemento
    \item Cálculo de matrices de transformación de coordenadas
    \item Obtención de matrices de rigidez globales
    \item Ensamblaje de la matriz de rigidez global del sistema
    \item Visualización gráfica de la estructura
\end{enumerate}

\subsection{Módulo de Procesamiento}

Este módulo resuelve el sistema de ecuaciones: 

\begin{enumerate}
    \item Ensamblaje del vector de fuerzas nodales
    \item Identificación de grados de libertad prescritos y libres
    \item Resolución del sistema para obtener desplazamientos
    \item Cálculo de tensiones en cada elemento
    \item Verificación del criterio de fallo (tensión > límite elástico)
    \item Cálculo de reacciones en los apoyos
\end{enumerate}

\subsection{Módulo de Postprocesamiento}

El postprocesamiento genera: 

\begin{enumerate}
    \item Visualización de la estructura deformada con escala amplificada
    \item Mapa de colores representando la distribución de tensiones
    \item Identificación visual de barras fallidas
\end{enumerate}

\subsection{Salida del Programa}

La ejecución del programa principal \texttt{main.m} genera la siguiente salida por consola y guarda los resultados en un archivo de texto (\texttt{output\_***.txt}):

\begin{lstlisting}[caption=Estructura de la salida del programa]
--- Paso 1: Lectura de input ---
Lectura de input completada

--- Paso 2: Preprocesamiento ---
Matriz de rigidez global (S): ...

--- Paso 3: Procesamiento ---
========================================
    RESULTADOS - NUDOS ***
========================================
--- Desplazamientos nodales (d) ---
...
--- Reacciones en los Apoyos ---
...

--- Paso 4: Postprocesamiento ---

--- Simulacion completada correctamente ---
Resultados guardados en: output_***.txt
\end{lstlisting}

Esta salida confirma que todos los módulos se ejecutan secuencialmente sin errores, validando la correcta implementación del programa. Los resultados detallados se exportan automáticamente al archivo \texttt{output\_***.txt}.

\newpage
\section{Desarrollo de los Apartados del Enunciado}

\subsection{Apartados a) y b): Representación e Identificación de la Estructura}

La estructura analizada es una cercha reticulada plana con:
\begin{itemize}
    \item 24 nodos
    \item 51 barras (3 tipos según sección transversal)
    \item 4 apoyos fijos (nodos 1, 2, 8, 9)
    \item Dimensiones: 50 m × 5 m
\end{itemize}

Las \textcolor{blue}{\underline{Figuras \ref{fig:estructura_biarticulado}}} y \textcolor{blue}{\underline{\ref{fig:estructura_rigido}}} muestran la representación gráfica con:
\begin{itemize}
    \item Identificación de todos los nodos (1-24)
    \item Elementos numerados (barras 1-51)
    \item Tres tipos de barra diferenciados por color según su sección transversal
\end{itemize}

\subsection{Apartado c) Desplazamientos Nodales}

\begin{table}[ht]
\centering
\begin{tabular}{@{}ccc@{}}
\toprule
Nodo & $u_x$ [m] & $u_y$ [m]\\
\midrule
1 & $0$ & $0$\\
2 & $0$ & $0$\\
3 & $-2.43\times10^{-5}$ & $-8.69\times10^{-3}$\\
4 & $-3.61\times10^{-5}$ & $-1.15\times10^{-2}$ \\
5 & $-4.14\times10^{-5}$ & $-1.31\times10^{-2}$ \\
6 & $-3.61\times10^{-5}$ & $-1.15\times10^{-2}$ \\
7 & $-2.43\times10^{-5}$ & $-8.69\times10^{-3}$\\
8 & $0$ & $0$\\
9 & $0$ & $0$\\
\bottomrule
\end{tabular}
\caption{Desplazamientos nodales en la banda de rodadura -- Nudos biarticulados}
\end{table}

\newpage
\subsection{Apartado d) Reacciones en Apoyos y Verificación del Equilibrio}

\begin{table}[ht]
\centering
\begin{tabular}{@{}ccc@{}}
\toprule
Nodo & $R_x$ [N] & $R_y$ [N] \\
\midrule
1 & $-4.5035\times10^{5}$ & $-2.3528\times10^{5}$ \\
2 & $9.8652\times10^{5}$ & $1.2353\times10^{6}$ \\
8 & $-9.8652\times10^{5}$ & $1.2353\times10^{6}$ \\
9 & $4.5035\times10^{5}$ & $-2.3528\times10^{5}$ \\
\midrule
\textbf{$\sum$} & $\approx 0$ & $2.0\times10^{6}$ \\
\bottomrule
\end{tabular}
\caption{Reacciones en los apoyos -- Nudos biarticulados}
\end{table}

\textbf{Verificación del equilibrio:}
\begin{align*}
\sum F_x &= 0 \text{ N} \quad \checkmark \\
\sum F_y &= 0 \text{ N} \quad \checkmark \\
\end{align*}

\subsection{Apartado e) Estructura Deformada con Escala de Colores}

La \textcolor{blue}{\underline{Figura \ref{fig:deformada_biarticulado}}} muestra:
\begin{itemize}
    \item Estructura deformada con escala amplificada para visualización
    \item Escala de colores representando la tensión axial en cada barra
    \item Barras en compresión (azul) y tracción (rojo)
\end{itemize}

\subsection{Apartado f) Identificación de Barra Crítica y Verificación de Fallo}

\textbf{Barras con mayores tensiones (top 5):}

\begin{table}[ht]
\centering
\begin{tabular}{@{}ccc@{}}
\toprule
Barra & Tensión [MPa] & Estado \\
\midrule
48 & $-55.62$ & Compresión \\
49 & $-55.62$ & Compresión \\
44 & $-46.41$ & Compresión \\
45 & $-46.41$ & Compresión \\
40 & $-28.08$ & Compresión \\
\bottomrule
\end{tabular}
\caption{Barras con mayores tensiones -- Nudos biarticulados}
\end{table}

\textbf{Barra con mayor tensión:} Barra 48 con $\sigma = -55.62$ MPa (compresión)

\textbf{Verificación de fallo:}
\begin{itemize}
    \item Límite elástico del material: $\sigma_y = 250$ MPa
    \item $|\sigma_{max}| = 55.62$ MPa $< 250$ MPa \quad \checkmark
    \item \textbf{Conclusión:} Ninguna barra falla
\end{itemize}

\subsection{Apartado g) Cálculo de Desplazamientos y Comparación}

\textbf{Resultados de desplazamientos nodales -- Nudos rígidos:}

\begin{table}[ht]
\centering
\begin{tabular}{@{}cccc@{}}
\toprule
Nodo & $u_x$ [m] & $u_y$ [m] & $\theta$ [rad]\\
\midrule
1 & $0$ & $0$ & $0$ \\
2 & $0$ & $0$ & $0$ \\
3 & $-2.43\times10^{-5}$ & $-8.67\times10^{-3}$ & $-1.2\times10^{-4}$\\
4 & $-3.60\times10^{-5}$ & $-1.15\times10^{-2}$ & $-1.5\times10^{-4}$ \\
5 & $-4.13\times10^{-5}$ & $-1.30\times10^{-2}$ & $-1.6\times10^{-4}$ \\
6 & $-3.60\times10^{-5}$ & $-1.15\times10^{-2}$ & $-1.5\times10^{-4}$ \\
7 & $-2.43\times10^{-5}$ & $-8.67\times10^{-3}$ & $-1.2\times10^{-4}$\\
8 & $0$ & $0$ & $0$\\
9 & $0$ & $0$ & $0$\\
\bottomrule
\end{tabular}
\caption{Desplazamientos nodales -- Nudos rígidos}
\end{table}


La reducción se debe a la rigidez adicional por resistencia a flexión, aunque el efecto es marginal dado que el comportamiento está dominado por esfuerzos axiales.

\newpage
\subsection{Apartado h) Reacciones en Apoyos y Verificación del Equilibrio}

\begin{table}[ht]
\centering
\begin{tabular}{@{}cccc@{}}
\toprule
Nodo & $R_x$ [N] & $R_y$ [N] & $M$ [N·m] \\
\midrule
1 & $-4.4821\times10^{5}$ & $-2.3491\times10^{5}$ & $-1.53\times10^{3}$ \\
2 & $9.8239\times10^{5}$ & $1.2349\times10^{6}$ & $1.1453\times10^{4}$ \\
8 & $-9.8239\times10^{5}$ & $1.2349\times10^{6}$ & $-1.1453\times10^{4}$ \\
9 & $4.4821\times10^{5}$ & $-2.3491\times10^{5}$ & $1.53\times10^{3}$ \\
\midrule
\textbf{$\sum$} & $\approx 0$ & $2.0\times10^{6}$ & $\approx 0$ \\
\bottomrule
\end{tabular}
\caption{Reacciones en los apoyos -- Nudos rígidos}
\end{table}

\textbf{Verificación del equilibrio:}
\begin{align*}
\sum F_x &\approx 0 \text{ N} \quad \checkmark \\
\sum F_y &= 2.0\times10^6 \text{ N} \quad \checkmark \\
\sum M &\approx 0 \text{ N·m} \quad \checkmark
\end{align*}

\subsection{Apartado i) Estructura Deformada con Escala de Colores}

La \textcolor{blue}{\underline{Figura \ref{fig:deformada_rigido}}} muestra:
\begin{itemize}
    \item Estructura deformada con escala amplificada
    \item Escala de colores para tensión total (axial + flexión)
    \item Distribución similar al modelo biarticulado con pequeñas variaciones
\end{itemize}

\subsection{Apartado j) Identificación de Barra Crítica y Verificación de Fallo}

\textbf{Barras con mayores tensiones -- Nudos rígidos:}

\begin{table}[ht]
\centering
\begin{tabular}{@{}ccc@{}}
\toprule
Barra & Tensión [MPa] & Estado \\
\midrule
48 & $-55.04$ & Compresión \\
49 & $-55.04$ & Compresión \\
44 & $-45.93$ & Compresión \\
45 & $-45.93$ & Compresión \\
40 & $-27.79$ & Compresión \\
\bottomrule
\end{tabular}
\caption{Barras con mayores tensiones -- Nudos rígidos}
\end{table}

\textbf{Barra con mayor tensión:} Barra 48 con $\sigma_{total} = -55.04$ MPa (compresión)

\textbf{Verificación de fallo:}
\begin{itemize}
    \item $|\sigma_{max}| = 55.04$ MPa $< 250$ MPa \quad \checkmark
    \item \textbf{Conclusión:} Ninguna barra falla
    \item La misma barra crítica que en el modelo biarticulado
\end{itemize}

\newpage
\section{Figuras de Resultados}

\subsection{Visualización de la Estructura -- Nudos Biarticulados}

\begin{figure}[htpb]
    \centering
    \includegraphics[width=0.75\linewidth]{\detokenize{Nudos biarticulados/figuras/estructura.png}}
    \caption{Representación gráfica de la estructura con nudos biarticulados. Los colores indican los tres tipos de sección transversal diferentes.}
    \label{fig:estructura_biarticulado}
\end{figure}

\subsection{Estructura Deformada y Tensiones -- Nudos Biarticulados}

\begin{figure}[htbp]
    \centering
    \includegraphics[width=0.75\linewidth]{\detokenize{Nudos biarticulados/figuras/deformada.png}}
    \caption{Estructura deformada con escala de colores para tensión axial -- Nudos biarticulados. Las barras en azul están a compresión y en rojo a tracción.}
    \label{fig:deformada_biarticulado}
\end{figure}

\newpage
\subsection{Visualización de la Estructura -- Nudos Rígidos}

\begin{figure}[htbp]
    \centering
    \includegraphics[width=0.75\linewidth]{\detokenize{Nudos Rígidos/figuras/estructura.png}}
    \caption{Representación gráfica de la estructura con nudos rígidos. Identificación completa de nodos y elementos.}
    \label{fig:estructura_rigido}
\end{figure}

\subsection{Estructura Deformada y Tensiones -- Nudos Rígidos}

\begin{figure}[htbp]
    \centering
    \includegraphics[width=0.75\linewidth]{\detokenize{Nudos Rígidos/figuras/deformada.png}}
    \caption{Estructura deformada con escala de colores para tensión total (axial + flexión) -- Nudos rígidos.}
    \label{fig:deformada_rigido}
\end{figure} 

\newpage
\section{Conclusiones}

\begin{enumerate}
    \item Se ha desarrollado exitosamente un programa modular en MATLAB para el análisis de estructuras reticuladas planas mediante MEF, implementando dos modelos: nudos biarticulados y nudos rígidos.
    
    \item Ambos modelos proporcionan resultados muy similares (diferencia $< 1.1\%$), confirmando que el comportamiento está dominado por esfuerzos axiales para este tipo de estructura.
    
    \item El modelo biarticulado es suficientemente preciso con ventaja computacional (33\% menos GDL).
    
    \item La estructura es segura bajo las cargas especificadas: tensión máxima representa solo el 26\% del límite elástico.
    
    \item Los resultados han sido validados mediante verificación de equilibrio estático y simetría.
    
\end{enumerate}
\end{document}