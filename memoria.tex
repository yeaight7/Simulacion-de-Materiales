\documentclass[11pt,a4paper]{article}

\usepackage[utf8]{inputenc}
\usepackage[T1]{fontenc}
\usepackage[spanish]{babel}
\usepackage{amsmath,amssymb}
\usepackage{graphicx}
\usepackage{booktabs}
\usepackage{siunitx}
\usepackage{geometry}
\usepackage{hyperref}

\geometry{margin=2.5cm}
\sisetup{output-decimal-marker = {,}}

\title{Memoria del trabajo de simulación de materiales}
\author{}
\date{}

\begin{document}
\maketitle

\section*{Resumen}
Este trabajo presenta la simulación de una estructura reticulada en dos hipótesis de conexión: nudos biarticulados (barras articuladas) y nudos rígidos (pórtico plano). Se implementa un flujo de preprocesado, procesado y postprocesado en MATLAB para ensamblar la matriz global de rigidez, resolver los desplazamientos, calcular tensiones y verificar el equilibrio global. Se comparan los resultados principales de ambos modelos y se discuten las diferencias observadas.

\section{Descripción del problema}
La estructura analizada está definida por un conjunto de nodos, barras, materiales, secciones y cargas verticales aplicadas en los nodos superiores. El modelo se resuelve en 2D con:
\begin{itemize}
    \item 24 nodos y 51 barras.
    \item 4 apoyos en los nodos 1, 2, 8 y 9.
    \item 3 materiales y 3 secciones distintas.
    \item Cargas verticales concentradas en los nodos 1 a 9, con un total de \SI{-2e6}{N}.
\end{itemize}

\section{Datos de entrada}
\subsection{Materiales}
\begin{table}[h]
\centering
\caption{Propiedades de materiales.}
\begin{tabular}{@{}ccc@{}}
\toprule
Material & $E$ (Pa) & $\sigma_y$ (Pa) \\
\midrule
1 & $2.00\times10^{11}$ & $2.50\times10^{8}$ \\
2 & $1.60\times10^{11}$ & $2.50\times10^{8}$ \\
3 & $1.60\times10^{11}$ & $2.10\times10^{8}$ \\
\bottomrule
\end{tabular}
\end{table}

\subsection{Secciones}
\begin{table}[h]
\centering
\caption{Propiedades geométricas de secciones.}
\begin{tabular}{@{}ccc@{}}
\toprule
Sección & $A$ (m$^2$) & $I$ (m$^4$) \\
\midrule
1 & 0.0150 & $2.80\times10^{-5}$ \\
2 & 0.0144 & $1.728\times10^{-5}$ \\
3 & 0.0254 & $5.153\times10^{-5}$ \\
\bottomrule
\end{tabular}
\end{table}

\subsection{Apoyos y cargas}
Los apoyos restringen los desplazamientos en $x$ e $y$ en los nodos 1, 2, 8 y 9. En el caso de nudos rígidos, se restringe también la rotación. Las cargas son verticales y se aplican en los nodos 1 a 9, con los valores (en N):
\begin{itemize}
    \item Nodos 1 y 9: $-125\,000$.
    \item Nodos 2 al 8: $-250\,000$.
\end{itemize}

\section{Metodología}
\subsection{Modelo con nudos biarticulados}
El modelo considera dos grados de libertad por nodo (desplazamientos $x$ e $y$). La matriz de rigidez local de cada barra se calcula con $k = EA/L$ y se transforma al sistema global mediante una matriz de rotación. La matriz global se ensambla sumando las contribuciones de cada barra y se resuelve el sistema reducido $S_{dd} d_d = p_d - S_{dp} d_p$.

Las tensiones axiales se obtienen como:
\begin{equation}
\sigma = E\frac{\Delta L}{L},
\end{equation}
siendo $\Delta L$ el alargamiento/ acortamiento calculado con los desplazamientos nodales. Se comprueba el límite elástico de cada sección para identificar fallos.

\subsection{Modelo con nudos rígidos}
El modelo con nudos rígidos incorpora tres grados de libertad por nodo (traslaciones $x$, $y$ y rotación $\theta$). Se emplea la matriz de rigidez local de viga de Euler-Bernoulli (6$\times$6), y se transforma al sistema global con una matriz de rotación 6$\times$6. La tensión total se aproxima como combinación de axial y flexión:
\begin{equation}
\sigma_{\text{total}} = \sigma_{\text{axial}} + \sigma_{\text{flex}},
\end{equation}
con $\sigma_{\text{axial}} = E\,\Delta L/L$ y $\sigma_{\text{flex}} = M c / I$.

\section{Resultados}
\subsection{Nudos biarticulados}
\begin{itemize}
    \item Desplazamiento máximo: nodo 5, $|u| = \SI{1.3055e-2}{m}$ (principalmente en $y$).
    \item Tensión máxima (valor absoluto): barra 48, $\sigma_{\max} = -\SI{5.562e7}{Pa}$.
    \item No se detectan barras fallidas.
\end{itemize}

\begin{table}[h]
\centering
\caption{Reacciones en apoyos (nudos biarticulados).}
\begin{tabular}{@{}cccc@{}}
\toprule
Nodo & $R_x$ (N) & $R_y$ (N) \\
\midrule
1 & $-4.5035\times10^{5}$ & $-2.3528\times10^{5}$ \\
2 & $9.8652\times10^{5}$ & $1.2353\times10^{6}$ \\
8 & $-9.8652\times10^{5}$ & $1.2353\times10^{6}$ \\
9 & $4.5035\times10^{5}$ & $-2.3528\times10^{5}$ \\
\bottomrule
\end{tabular}
\end{table}

\begin{figure}[h]
    \centering
    \includegraphics[width=0.8\linewidth]{\detokenize{Nudos biarticulados/figuras/estructura.png}}
    \caption{Estructura con nudos biarticulados.}
\end{figure}

\begin{figure}[h]
    \centering
    \includegraphics[width=0.8\linewidth]{\detokenize{Nudos biarticulados/figuras/deformada.png}}
    \caption{Deformada y distribución de tensiones (nudos biarticulados).}
\end{figure}

\subsection{Nudos rígidos}
\begin{itemize}
    \item Desplazamiento máximo: nodo 5, $|u| = \SI{1.3019e-2}{m}$ (principalmente en $y$).
    \item Tensión máxima (valor absoluto): barra 48, $\sigma_{\max} = -\SI{5.504e7}{Pa}$.
    \item No se detectan barras fallidas.
\end{itemize}

\begin{table}[h]
\centering
\caption{Reacciones en apoyos (nudos rígidos).}
\begin{tabular}{@{}ccccc@{}}
\toprule
Nodo & $R_x$ (N) & $R_y$ (N) & $M$ (N·m) \\
\midrule
1 & $-4.4821\times10^{5}$ & $-2.3491\times10^{5}$ & $-1.53\times10^{3}$ \\
2 & $9.8239\times10^{5}$ & $1.2349\times10^{6}$ & $1.1453\times10^{4}$ \\
8 & $-9.8239\times10^{5}$ & $1.2349\times10^{6}$ & $-1.1453\times10^{4}$ \\
9 & $4.4821\times10^{5}$ & $-2.3491\times10^{5}$ & $1.53\times10^{3}$ \\
\bottomrule
\end{tabular}
\end{table}

\begin{figure}[h]
    \centering
    \includegraphics[width=0.8\linewidth]{\detokenize{Nudos Rígidos/figuras/estructura.png}}
    \caption{Estructura con nudos rígidos.}
\end{figure}

\begin{figure}[h]
    \centering
    \includegraphics[width=0.8\linewidth]{\detokenize{Nudos Rígidos/figuras/deformada.png}}
    \caption{Deformada y distribución de tensiones (nudos rígidos).}
\end{figure}

\section{Comparación y discusión}
\begin{itemize}
    \item Los desplazamientos máximos son muy similares en ambos casos, con una ligera reducción al considerar rigidez rotacional.
    \item La tensión máxima (compresión) también es del mismo orden, con una reducción cercana al 1\% en el modelo con nudos rígidos.
    \item Las reacciones muestran un equilibrio global adecuado: la suma de reacciones en $y$ coincide con la carga total aplicada y la suma de reacciones en $x$ es prácticamente nula.
\end{itemize}

\section{Conclusiones}
El análisis confirma que, para este caso de carga, la incorporación de rigidez rotacional modifica ligeramente los resultados sin alterar el comportamiento global. La estructura trabaja en régimen elástico en ambos modelos y no se identifican fallos. Los resultados obtenidos permiten validar el flujo de simulación y sirven como base para comparar hipótesis de conexión en estructuras reticuladas.

\end{document}
