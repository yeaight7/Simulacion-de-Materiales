\documentclass[12pt,a4paper]{article}

\usepackage[spanish]{babel}
\usepackage[T1]{fontenc}
\usepackage[utf8]{inputenc}
\usepackage{lmodern}

\usepackage{geometry}
\geometry{left=25mm,right=25mm,top=25mm,bottom=25mm}

\usepackage{graphicx}
\usepackage{float}
\usepackage{booktabs}
\usepackage{longtable}
\usepackage{amsmath}
\usepackage{siunitx}
\usepackage{hyperref}

\sisetup{
  output-decimal-marker = {.},
  group-separator = {\,},
  group-minimum-digits = 4
}

\title{Trabajo de curso: Puente biarticulado (pequeños desplazamientos)}
\author{\textbf{Nombre Apellidos}\\Simulación de Materiales y Estructuras\\Escuela Técnica Superior de Ingeniería, Universidad Loyola Andalucía}
\date{\today}

\begin{document}
\maketitle
\tableofcontents
\bigskip

\section{Introducción}
En este trabajo se analiza la estructura asignada (modelo de un puente) mediante el \textit{método matricial de la rigidez}.
La geometría global está definida por $L=\SI{50}{m}$ y $H=\SI{5}{m}$, y la carga total aplicada es una fuerza vertical descendente de $\SI{2}{MN}$ repartida sobre los nodos de la banda de rodadura.
En esta primera parte se considera que \textbf{todas las barras son biarticuladas} y se asume la \textbf{hipótesis de pequeños desplazamientos}.

\section{Datos del problema}
\subsection{Tipos de barra}
La estructura utiliza tres tipos de barra (material y sección), de acuerdo con el enunciado. En el modelo de nudos biarticulados solo se emplean $E$ y el área $A$ (los momentos de inercia $I$ quedan reservados para el caso de nudos rígidos).

\begin{table}[H]
\centering
\caption{Propiedades de los tres tipos de barra.}
\label{tab:tipos}
\begin{tabular}{lcccc}
\toprule
Tipo & Sección & $A$ [m$^2$] & $E$ [GPa] & $S_y$ [MPa] \\
\midrule
1 & Rectangular $15\times 10$ cm & 0.0150 & 200 & 250 \\
2 & Cuadrada $12\times 12$ cm    & 0.0144 & 160 & 250 \\
3 & Circular $D=180$ mm          & 0.0254 & 160 & 210 \\
\bottomrule
\end{tabular}
\end{table}

\subsection{Apoyos y cargas}
Los apoyos se sitúan en los nodos 1, 2, 8 y 9, restringiendo los dos grados de libertad $(u_x,u_y)$ en dichos nodos, de forma coherente con el archivo de entrada.
La carga total es $-\SI{2}{MN}$ en dirección vertical, aplicada en los nodos de la banda de rodadura (nodos 1--9) con el reparto:
\[
F_{y,1}=F_{y,9}=-\SI{125}{kN},\qquad
F_{y,2}=\dots=F_{y,8}=-\SI{250}{kN}.
\]

\section{Representación inicial}
\subsection*{(a) Representar la estructura indicando los tres tipos de barra}
\subsection*{(b) Identificar los nodos de la estructura}
En la Figura~\ref{fig:estructura} se representa la estructura con los tres tipos de barra (colores) y se numeran los nodos. Para identificar los elementos de forma no ambigua, se incluye la tabla de conectividad en el Apéndice~\ref{ap:conectividad}.

\begin{figure}[H]
\centering
\includegraphics[width=\linewidth]{fig_estructura.pdf}
\caption{Estructura inicial: tipos de barra (colores), numeración de nodos, apoyos y cargas aplicadas.}
\label{fig:estructura}
\end{figure}

\section{Barras biarticuladas y pequeños desplazamientos}
\subsection{Metodología de cálculo (resumen)}
Se ha implementado un flujo de cálculo modular (lectura de datos, preprocesado, procesado y postprocesado) basado en el método matricial:
\begin{itemize}
  \item Para cada barra $e$ se calcula su longitud $L_e$ y su orientación $(c=\cos\theta,\, s=\sin\theta)$.
  \item Se construye la matriz de rigidez local axial $k_e=\frac{E_e A_e}{L_e} \begin{bmatrix} 1 & -1\\ -1 & 1\end{bmatrix}$ y se transforma a coordenadas globales mediante una matriz de giro $T_e$.
  \item Se ensambla la matriz global $S$ en los grados de libertad $(u_x,u_y)$ de cada nodo.
  \item Se aplican las condiciones de contorno particionando en GDL libres y restringidos, y se resuelve el sistema $S_{dd}\,\mathbf{d}_d=\mathbf{p}_d$.
\end{itemize}

\subsection*{(c) Desplazamientos nodales}
En la Tabla~\ref{tab:desplazamientos} se muestran los desplazamientos nodales resultantes.
El máximo desplazamiento vertical se obtiene en el nodo 5, con $u_y=\SI{-13.055}{mm}$ (descenso).
El máximo desplazamiento horizontal en valor absoluto aparece en los nodos 19, 23, con $|u_x|=\SI{2.852}{mm}$.

\begin{longtable}{rrrrr}
\caption{Coordenadas y desplazamientos nodales (modelo biarticulado).}
\label{tab:desplazamientos}\\
\toprule
Nodo & $x$ [m] & $y$ [m] & $u_x$ [mm] & $u_y$ [mm] \\
\midrule
\endfirsthead
\toprule
Nodo & $x$ [m] & $y$ [m] & $u_x$ [mm] & $u_y$ [mm] \\
\midrule
\endhead
1 & 0.000 & 0.000 & 0.000 & 0.000 \\
2 & 6.250 & 0.000 & 0.000 & 0.000 \\
3 & 12.500 & 0.000 & -0.927 & -6.339 \\
4 & 18.750 & 0.000 & -0.802 & -11.141 \\
5 & 25.000 & 0.000 & -0.000 & -13.055 \\
6 & 31.250 & 0.000 & 0.802 & -11.141 \\
7 & 37.500 & 0.000 & 0.927 & -6.339 \\
8 & 43.750 & 0.000 & 0.000 & 0.000 \\
9 & 50.000 & 0.000 & 0.000 & 0.000 \\
10 & 3.125 & 2.500 & 0.364 & 0.455 \\
11 & 9.375 & 2.500 & 1.441 & -2.895 \\
12 & 15.625 & 2.500 & 0.678 & -8.880 \\
13 & 21.875 & 2.500 & 0.239 & -12.687 \\
14 & 28.125 & 2.500 & -0.239 & -12.687 \\
15 & 34.375 & 2.500 & -0.678 & -8.880 \\
16 & 40.625 & 2.500 & -1.441 & -2.895 \\
17 & 46.875 & 2.500 & -0.364 & 0.455 \\
18 & 6.250 & 5.000 & 2.413 & -1.198 \\
19 & 12.500 & 5.000 & 2.852 & -5.752 \\
20 & 18.750 & 5.000 & 1.739 & -10.741 \\
21 & 25.000 & 5.000 & 0.000 & -12.634 \\
22 & 31.250 & 5.000 & -1.739 & -10.741 \\
23 & 37.500 & 5.000 & -2.852 & -5.752 \\
24 & 43.750 & 5.000 & -2.413 & -1.198 \\
\bottomrule
\end{longtable}

\subsection*{(d) Reacciones en los apoyos y comprobación de equilibrio}
Las reacciones se calculan como $r = S\,d - p$. En la Tabla~\ref{tab:reacciones} se recogen las reacciones en los apoyos.
La comprobación global de equilibrio (suma de fuerzas externas más reacciones) conduce a un residuo numérico del orden de $10^{-8}$~N, atribuible a redondeo.

\begin{table}[H]
\centering
\caption{Reacciones en los apoyos (signo según ejes globales).}
\label{tab:reacciones}
\begin{tabular}{rcc}
\toprule
Nodo & $R_x$ [kN] & $R_y$ [kN] \\
\midrule
1 & -450.353 & -235.283 \\
2 & 986.517 & 1235.283 \\
8 & -986.517 & 1235.283 \\
9 & 450.353 & -235.283 \\
\bottomrule
\end{tabular}
\end{table}

\subsection*{(e) Estructura deformada con escala de colores para la tensión}
Las tensiones axiales se obtienen a partir de la deformación unitaria $\varepsilon_e \approx \Delta L_e/L_e$ y $\sigma_e = E_e\,\varepsilon_e$. 
La Figura~\ref{fig:tensiones} representa la estructura deformada (escala 100 para visualización) coloreada según $\sigma$.

\begin{figure}[H]
\centering
\includegraphics[width=\linewidth]{fig_tensiones.pdf}
\caption{Estructura deformada (escala 100) y distribución de tensiones.}
\label{fig:tensiones}
\end{figure}

\subsection*{(f) Barra con mayor tensión y comprobación de fallo}
La máxima tensión en valor absoluto es $|\sigma_{\max}|=\SI{55.624}{MPa}$, y se localiza en los elementos 48, 49 (compresión). 
Dado que $|\sigma_{\max}| < S_y$ para los tres tipos de barra, \textbf{no se detecta fallo} según el criterio elástico ($|\sigma|>S_y$).
A modo de resumen, la Tabla~\ref{tab:top} muestra los diez elementos más solicitados (por $|\sigma|$).

\begin{table}[H]
\centering
\caption{Elementos más solicitados (ordenados por $|\sigma|$).}
\label{tab:top}
\begin{tabular}{rrrrrrr}
\toprule
Elem. & $i$ & $j$ & Tipo & $L$ [m] & $\sigma$ [MPa] & $N$ [kN] \\
\midrule
49 & 21 & 22 & 1 & 6.250 & -55.624 & -834.358 \\
48 & 20 & 21 & 1 & 6.250 & -55.624 & -834.358 \\
11 & 2 & 18 & 2 & 5.000 & -38.309 & -551.644 \\
29 & 8 & 24 & 2 & 5.000 & -38.309 & -551.644 \\
50 & 22 & 23 & 1 & 6.250 & -35.567 & -533.502 \\
47 & 19 & 20 & 1 & 6.250 & -35.567 & -533.502 \\
7 & 7 & 8 & 1 & 6.250 & -29.551 & -443.261 \\
2 & 2 & 3 & 1 & 6.250 & -29.551 & -443.261 \\
34 & 11 & 19 & 3 & 4.002 & -27.264 & -692.502 \\
43 & 16 & 23 & 3 & 4.002 & -27.264 & -692.502 \\
\bottomrule
\end{tabular}
\end{table}

\section{Conclusiones (parte biarticulada)}
El modelo biarticulado predice un descenso máximo del tablero del orden de $\SI{13.06}{mm}$ bajo la carga distribuida de $\SI{2}{MN}$.
Las tensiones permanecen por debajo de los límites elásticos indicados, por lo que la estructura no presenta fallo en régimen elástico para este caso de carga.

\appendix
\section{Conectividad de elementos}
\label{ap:conectividad}
La Tabla~\ref{tab:conectividad} define la conectividad nodal y el tipo de barra de cada elemento ($\text{Tipo}=1,2,3$).

\begin{longtable}{rrrrr}
\caption{Conectividad de elementos y tipo de barra.}
\label{tab:conectividad}\\
\toprule
Elemento & $i$ & $j$ & Tipo & $L$ [m] \\
\midrule
\endfirsthead
\toprule
Elemento & $i$ & $j$ & Tipo & $L$ [m] \\
\midrule
\endhead
1 & 1 & 2 & 1 & 6.250 \\
2 & 2 & 3 & 1 & 6.250 \\
3 & 3 & 4 & 1 & 6.250 \\
4 & 4 & 5 & 1 & 6.250 \\
5 & 5 & 6 & 1 & 6.250 \\
6 & 6 & 7 & 1 & 6.250 \\
7 & 7 & 8 & 1 & 6.250 \\
8 & 8 & 9 & 1 & 6.250 \\
9 & 1 & 10 & 3 & 4.002 \\
10 & 2 & 10 & 3 & 4.002 \\
11 & 2 & 18 & 2 & 5.000 \\
12 & 2 & 11 & 3 & 4.002 \\
13 & 3 & 11 & 3 & 4.002 \\
14 & 3 & 19 & 2 & 5.000 \\
15 & 3 & 12 & 3 & 4.002 \\
16 & 4 & 12 & 3 & 4.002 \\
17 & 4 & 20 & 2 & 5.000 \\
18 & 4 & 13 & 3 & 4.002 \\
19 & 5 & 13 & 3 & 4.002 \\
20 & 5 & 21 & 2 & 5.000 \\
21 & 5 & 14 & 3 & 4.002 \\
22 & 6 & 14 & 3 & 4.002 \\
23 & 6 & 22 & 2 & 5.000 \\
24 & 6 & 15 & 3 & 4.002 \\
25 & 7 & 15 & 3 & 4.002 \\
26 & 7 & 23 & 2 & 5.000 \\
27 & 7 & 16 & 3 & 4.002 \\
28 & 8 & 16 & 3 & 4.002 \\
29 & 8 & 24 & 2 & 5.000 \\
30 & 8 & 17 & 3 & 4.002 \\
31 & 9 & 17 & 3 & 4.002 \\
32 & 10 & 18 & 3 & 4.002 \\
33 & 11 & 18 & 3 & 4.002 \\
34 & 11 & 19 & 3 & 4.002 \\
35 & 12 & 19 & 3 & 4.002 \\
36 & 12 & 20 & 3 & 4.002 \\
37 & 13 & 20 & 3 & 4.002 \\
38 & 13 & 21 & 3 & 4.002 \\
39 & 14 & 21 & 3 & 4.002 \\
40 & 14 & 22 & 3 & 4.002 \\
41 & 15 & 22 & 3 & 4.002 \\
42 & 15 & 23 & 3 & 4.002 \\
43 & 16 & 23 & 3 & 4.002 \\
44 & 16 & 24 & 3 & 4.002 \\
45 & 17 & 24 & 3 & 4.002 \\
46 & 18 & 19 & 1 & 6.250 \\
47 & 19 & 20 & 1 & 6.250 \\
48 & 20 & 21 & 1 & 6.250 \\
49 & 21 & 22 & 1 & 6.250 \\
50 & 22 & 23 & 1 & 6.250 \\
51 & 23 & 24 & 1 & 6.250 \\
\bottomrule
\end{longtable}

\end{document}